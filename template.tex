% Options for packages loaded elsewhere
% Options for packages loaded elsewhere
\PassOptionsToPackage{unicode}{hyperref}
\PassOptionsToPackage{hyphens}{url}
\PassOptionsToPackage{dvipsnames,svgnames,x11names}{xcolor}
%
\documentclass[
]{rrling}
\usepackage{xcolor}
\usepackage{amsmath,amssymb}
\setcounter{secnumdepth}{5}
\usepackage{iftex}
\ifPDFTeX
  \usepackage[T1]{fontenc}
  \usepackage[utf8]{inputenc}
  \usepackage{textcomp} % provide euro and other symbols
\else % if luatex or xetex
  \usepackage{unicode-math} % this also loads fontspec
  \defaultfontfeatures{Scale=MatchLowercase}
  \defaultfontfeatures[\rmfamily]{Ligatures=TeX,Scale=1}
\fi
\usepackage{lmodern}
\ifPDFTeX\else
  % xetex/luatex font selection
\fi
% Use upquote if available, for straight quotes in verbatim environments
\IfFileExists{upquote.sty}{\usepackage{upquote}}{}
\IfFileExists{microtype.sty}{% use microtype if available
  \usepackage[]{microtype}
  \UseMicrotypeSet[protrusion]{basicmath} % disable protrusion for tt fonts
}{}
\makeatletter
\@ifundefined{KOMAClassName}{% if non-KOMA class
  \IfFileExists{parskip.sty}{%
    \usepackage{parskip}
  }{% else
    \setlength{\parindent}{0pt}
    \setlength{\parskip}{6pt plus 2pt minus 1pt}}
}{% if KOMA class
  \KOMAoptions{parskip=half}}
\makeatother
% Make \paragraph and \subparagraph free-standing
\makeatletter
\ifx\paragraph\undefined\else
  \let\oldparagraph\paragraph
  \renewcommand{\paragraph}{
    \@ifstar
      \xxxParagraphStar
      \xxxParagraphNoStar
  }
  \newcommand{\xxxParagraphStar}[1]{\oldparagraph*{#1}\mbox{}}
  \newcommand{\xxxParagraphNoStar}[1]{\oldparagraph{#1}\mbox{}}
\fi
\ifx\subparagraph\undefined\else
  \let\oldsubparagraph\subparagraph
  \renewcommand{\subparagraph}{
    \@ifstar
      \xxxSubParagraphStar
      \xxxSubParagraphNoStar
  }
  \newcommand{\xxxSubParagraphStar}[1]{\oldsubparagraph*{#1}\mbox{}}
  \newcommand{\xxxSubParagraphNoStar}[1]{\oldsubparagraph{#1}\mbox{}}
\fi
\makeatother


\usepackage{longtable,booktabs,array}
\usepackage{calc} % for calculating minipage widths
% Correct order of tables after \paragraph or \subparagraph
\usepackage{etoolbox}
\makeatletter
\patchcmd\longtable{\par}{\if@noskipsec\mbox{}\fi\par}{}{}
\makeatother
% Allow footnotes in longtable head/foot
\IfFileExists{footnotehyper.sty}{\usepackage{footnotehyper}}{\usepackage{footnote}}
\makesavenoteenv{longtable}
\usepackage{graphicx}
\makeatletter
\newsavebox\pandoc@box
\newcommand*\pandocbounded[1]{% scales image to fit in text height/width
  \sbox\pandoc@box{#1}%
  \Gscale@div\@tempa{\textheight}{\dimexpr\ht\pandoc@box+\dp\pandoc@box\relax}%
  \Gscale@div\@tempb{\linewidth}{\wd\pandoc@box}%
  \ifdim\@tempb\p@<\@tempa\p@\let\@tempa\@tempb\fi% select the smaller of both
  \ifdim\@tempa\p@<\p@\scalebox{\@tempa}{\usebox\pandoc@box}%
  \else\usebox{\pandoc@box}%
  \fi%
}
% Set default figure placement to htbp
\def\fps@figure{htbp}
\makeatother


% definitions for citeproc citations
\NewDocumentCommand\citeproctext{}{}
\NewDocumentCommand\citeproc{mm}{%
  \begingroup\def\citeproctext{#2}\cite{#1}\endgroup}
\makeatletter
 % allow citations to break across lines
 \let\@cite@ofmt\@firstofone
 % avoid brackets around text for \cite:
 \def\@biblabel#1{}
 \def\@cite#1#2{{#1\if@tempswa , #2\fi}}
\makeatother
\newlength{\cslhangindent}
\setlength{\cslhangindent}{1.5em}
\newlength{\csllabelwidth}
\setlength{\csllabelwidth}{3em}
\newenvironment{CSLReferences}[2] % #1 hanging-indent, #2 entry-spacing
 {\begin{list}{}{%
  \setlength{\itemindent}{0pt}
  \setlength{\leftmargin}{0pt}
  \setlength{\parsep}{0pt}
  % turn on hanging indent if param 1 is 1
  \ifodd #1
   \setlength{\leftmargin}{\cslhangindent}
   \setlength{\itemindent}{-1\cslhangindent}
  \fi
  % set entry spacing
  \setlength{\itemsep}{#2\baselineskip}}}
 {\end{list}}
\usepackage{calc}
\newcommand{\CSLBlock}[1]{\hfill\break\parbox[t]{\linewidth}{\strut\ignorespaces#1\strut}}
\newcommand{\CSLLeftMargin}[1]{\parbox[t]{\csllabelwidth}{\strut#1\strut}}
\newcommand{\CSLRightInline}[1]{\parbox[t]{\linewidth - \csllabelwidth}{\strut#1\strut}}
\newcommand{\CSLIndent}[1]{\hspace{\cslhangindent}#1}



\setlength{\emergencystretch}{3em} % prevent overfull lines

\providecommand{\tightlist}{%
  \setlength{\itemsep}{0pt}\setlength{\parskip}{0pt}}



 


% TODO: Add custom LaTeX header directives here

\setmainfont{Source Sans Pro}
\setsansfont{Source Sans Pro}
\makeatletter
\@ifpackageloaded{caption}{}{\usepackage{caption}}
\AtBeginDocument{%
\ifdefined\contentsname
  \renewcommand*\contentsname{Table of contents}
\else
  \newcommand\contentsname{Table of contents}
\fi
\ifdefined\listfigurename
  \renewcommand*\listfigurename{List of Figures}
\else
  \newcommand\listfigurename{List of Figures}
\fi
\ifdefined\listtablename
  \renewcommand*\listtablename{List of Tables}
\else
  \newcommand\listtablename{List of Tables}
\fi
\ifdefined\figurename
  \renewcommand*\figurename{Figure}
\else
  \newcommand\figurename{Figure}
\fi
\ifdefined\tablename
  \renewcommand*\tablename{Table}
\else
  \newcommand\tablename{Table}
\fi
}
\@ifpackageloaded{float}{}{\usepackage{float}}
\floatstyle{ruled}
\@ifundefined{c@chapter}{\newfloat{codelisting}{h}{lop}}{\newfloat{codelisting}{h}{lop}[chapter]}
\floatname{codelisting}{Listing}
\newcommand*\listoflistings{\listof{codelisting}{List of Listings}}
\makeatother
\makeatletter
\makeatother
\makeatletter
\@ifpackageloaded{caption}{}{\usepackage{caption}}
\@ifpackageloaded{subcaption}{}{\usepackage{subcaption}}
\makeatother
\usepackage{bookmark}
\IfFileExists{xurl.sty}{\usepackage{xurl}}{} % add URL line breaks if available
\urlstyle{same}
\hypersetup{
  pdftitle={Registered Reports in Linguistics: Quarto template},
  pdfauthor={Stefano Coretta; Jessica Hampton},
  pdfkeywords={template, demo},
  colorlinks=true,
  linkcolor={blue},
  filecolor={Maroon},
  citecolor={Blue},
  urlcolor={Blue},
  pdfcreator={LaTeX via pandoc}}



\title{Registered Reports in Linguistics: Quarto template}
\addauthor{Stefano Coretta}{0000-0001-9627-5532}{1}\addauthor{Jessica
Hampton}{0000-0001-6871-2846}{2}
\addaffiliation{1}{University of Edinburgh}\addaffiliation{2}{University
of of Liverpool}
\date{}
\abstract{This abstract serves as a generic template to guide authors in
crafting effective summaries of their academic work. An ideal abstract
should succinctly present the purpose of the study, the methodology
employed, the key findings, and the principal conclusions. Begin by
clearly stating the research question or objective, ensuring it reflects
the core focus of the article. Follow with a brief description of the
methods used, highlighting the design, data sources, and analytical
techniques where applicable. Then, summarize the most significant
results, avoiding excessive detail or technical jargon. Conclude with a
statement on the broader implications or potential applications of the
findings. The abstract should be self-contained, written in the past
tense (except for conclusions or implications, which may be in the
present), and limited to 200--250 words. Avoid citations, abbreviations,
or references to tables and figures. Emphasize clarity, precision, and
coherence to ensure accessibility to a wide readership, including those
outside the immediate disciplinary field.}
\keywords{template, demo}
\rescompendium{https://www.doi.org/10.1000/182}

\newfontfamily{\defaultfont}[Mapping=tex-text]{Source Sans Pro}
\newfontfamily{\CJKfont}[ItalicFont={Source Sans Pro Italic},BoldFont={Source Sans Pro Semibold},BoldItalicFont={Source Sans Pro Semibold Italic}]{Yuppy
SC}
\newfontfamily{\Devanagarifont}[ItalicFont={Source Sans Pro Italic},BoldFont={Source Sans Pro Semibold},BoldItalicFont={Source Sans Pro Semibold Italic}]{Noto
Serif Devanagari}
\usepackage[Latin,CJK,Devanagari,]{ucharclasses}
\setTransitionsForLatin{\defaultfont}{}
\setTransitionsForCJK{\CJKfont}{}
\setTransitionsForDevanagari{\Devanagarifont}{}

\begin{document}
\maketitle


\section*{The RRLing templates}\label{the-rrling-templates}
\addcontentsline{toc}{section}{The RRLing templates}

All submissions to Registered Reports in Linguistics (RRLing) must use
this Quarto template (\url{https://github.com/stefanocoretta/rrling}) or
the companion LaTeX template
(\url{https://www.overleaf.com/read/bbcpjgrnvpny\#8ae823}). We strongly
encourage authors to use the Quarto template (especially if their study
involves any type of statistical analysis). If you are new to both LaTeX
and Quarto, then we suggest to use Quarto with RStudio/Positron for a
smoother writing experience if your study has computational code (so
that you can embed the code in the manuscript), or to use LaTeX on
Overleaf if you don't need support for computational code in the
manuscript.

\textbf{Both Quarto and LaTeX documents can be rendered to PDF and this
will be the final format used for accepted papers. Note that we will not
accept papers that are not written with one of the templates.} This is
because we don't have resources for typesetting accepted articles and by
using the official templates, no further special typesetting is needed.

\subsection*{LaTeX}\label{latex}
\addcontentsline{toc}{subsection}{LaTeX}

If you are new to LaTeX, you can learn the basics here:
\url{https://www.overleaf.com/learn/latex/Learn_LaTeX_in_30_minutes}.
The LaTeX template can be found at
\url{https://www.overleaf.com/read/bbcpjgrnvpny\#8ae823} and used online
on Overleaf (\url{https://www.overleaf.com}) or can be downloaded to
your computer. The template is designed to showcase the basic structure
of an article submitted to RRLing, including examples of specific
typesetting requirements.

\subsection*{Quarto}\label{quarto}
\addcontentsline{toc}{subsection}{Quarto}

Quarto (\url{https://quarto.org}) is a new publishing system that allows
you to write text and code in a single document. The document can be
rendered to a variety of formats, including the PDF format required for
submission to RRLing.

A Quarto document can dynamically include in the rendered document the
output of computational code, such as summary tables and plots, with the
advantage that, if the code changes, the new output is automatically
reflected in the document when rendered. An introduction to using Quarto
can be found here: \url{https://quarto.org/docs/get-started/}. This
Quarto template is available here:
\url{https://github.com/stefanocoretta/rrling}.

The Quarto template for submissions to RRLing can be found at
\url{https://github.com/stefanocoretta/rrling}. We recommend using
RStudio/Positron to write with the Quarto template, but any other
workflow you already use will do.

\section*{Article structure}\label{article-structure}
\addcontentsline{toc}{section}{Article structure}

All articles submitted to RRLing should have the following structure:

\begin{enumerate}
\def\labelenumi{\arabic{enumi}.}
\item
  Introduction
\item
  Methods
\item
  Results
\item
  Discussion
\item
  Conclusion
\item
  (unnumbered) CRediT roles
\item
  (unnumbered) Acknowledgements (optional)
\item
  (unnumbered) Open Research statement
\item
  (unnumbered) Conflict of interest
\item
  (unnumbered) References
\end{enumerate}

Note that in the Stage 1 manuscript, only the following sections are
necessary: Introduction, Methods, CRediT roles, Open Research statement,
Conflict of interest. You will not be able to make changes to the
Introduction and Methods section in the Stage 2 manuscript (barring
minor typographical or grammatical edits).

You should also provide an abstract (250 words max), keywords and a link
to the Research Compendium of the study (this can be a URL or a DOI).
The Research Compendium is an online repository containing all the
materials of the study, including but not limited to, all stimuli, data,
code, figures, and so on.

The following sections mirror the sections that articles submitted to
RRLing should have. Each section provides information on what the
section should include. Within each section, it is up to the author to
include subsections (no specific requirements are in place in this case,
but we encourage authors to make use of subsections to aid readers).

\section{Introduction}\label{introduction}

The Introduction should include a general overview of the research
topic, followed by a review of existing relevant literature to summarise
existing knowledge. It is not expected that authors include all the
existing literature, under the assumption that due diligence was carried
out. We rather encourage authors to point readers to existing literature
reviews, if they exist and are relevant. Suggested subsections include a
``Background'' or ``Literature Review'' subsection and a ``Research
Questions/Hypotheses'' subsection.

\subsection*{Typography}\label{typography}
\addcontentsline{toc}{subsection}{Typography}

\subsubsection*{Typefaces}\label{typefaces}
\addcontentsline{toc}{subsubsection}{Typefaces}

You can mix scripts thanks to the LaTeX package \texttt{ucharclasses}.
For example, this is Chinese 普通話 \emph{pǔtōnghuà} and this is Sankrit
देवनागरी \emph{devanāgarī}. The Latin font is Source Sans Pro and should
not be changed. Source Sans Pro also includes graphs from the
International Phonetic Alphabet: {[}ˌɪntəˈnæʃənəl fəˈnɛtɪk ˈælfəbɛt{]}.

\subsubsection*{Tables and figures}\label{tables-and-figures}
\addcontentsline{toc}{subsubsection}{Tables and figures}

You can include tables like Table~\ref{tbl-letters}. The caption will be
placed above the table. You can also include figures, like
Figure~\ref{fig-example}. Place the caption below the figure. Note that
table and figure positioning is managed by LaTeX: do not force placement
with LaTeXplacement options. You can use sub-figures and a two-column
layout if it uses the space better.

\begin{longtable}[]{@{}lll@{}}
\caption{My Caption}\label{tbl-letters}\tabularnewline
\toprule\noalign{}
Col1 & Col2 & Col3 \\
\midrule\noalign{}
\endfirsthead
\toprule\noalign{}
Col1 & Col2 & Col3 \\
\midrule\noalign{}
\endhead
\bottomrule\noalign{}
\endlastfoot
A & B & C \\
E & F & G \\
A & G & G \\
\end{longtable}

\begin{figure}

\centering{

\includegraphics[width=0.25\linewidth,height=\textheight,keepaspectratio]{rrling-logo-bg.png}

}

\caption{\label{fig-example}The Registered Reports in Linguistics logo.}

\end{figure}%

\subsubsection*{Linguistic examples and interlinear
glosses}\label{linguistic-examples-and-interlinear-glosses}
\addcontentsline{toc}{subsubsection}{Linguistic examples and interlinear
glosses}

If you need to include linguistic examples or interlinear glosses,
please use the interlinear Quarto extension:
\url{https://github.com/stefanocoretta/interlinear}. If the extension is
too limited for your glossing needs, we recommend to use the LaTeX
template instead so that you can exploit the full capabilities of the
LaTeX package expex.

\subsubsection*{Bibliography}\label{bibliography}
\addcontentsline{toc}{subsubsection}{Bibliography}

The \texttt{bibliography.bib} file has a few example entries in BiBTeX
format. To cite use the
\href{https://quarto.org/docs/authoring/citations.html}{Quarto citation
syntax}: for example Alfarano (2021). You can also include both author
and year within the parentheses (Croft 2002). You can specify pages and
custom text (see Jakobson, Fant \& Halle 1951: 47--52; and Lindsey
2017).

Bibliographical entries will be added to the (unnumbered)
\texttt{References} section at the end of the article.

\section{Methods}\label{methods}

The Methods section should provide detailed information about the
methods that will be employed in the study after In Principle Acceptance
of the Stage 1 Registered Report manuscript. There should be enough
details for reviewers to be able to reproduce exactly the same pipeline
employed in the study (at Stage 1) and for reviewers and independent
researchers to obtain the same results which will be reported on the
Stage 2 manuscript. This level of detail will also make replication
attempts more straightforward.

Note that, according to principles of Open Research which are mandated
for submission to this Journal, you must ensure that all materials
belonging to the Methods section at Stage 1 are available, including
e.g.~task-related files, programming code, questionnaires, interview
prompts, deductive models for thematic analysis, pre-existing corpora
etc.

\section{Results}\label{results}

After your Stage 1 manuscript has been In Principle Accepted and you
have conducted your study, this section in the Stage 2 manuscript should
report the results of the methods described in the previous session. The
outcome of all registered analysis from the Stage 1 manuscript must be
reported in this section. Further, non-registered analyses, can be
included under a separate subsection and clearly flagged as
non-registered (for example in a subsection called ``Non-registered
analyses'').

As stated above, Open Research principle are mandated for submission to
this Journal, so all materials that have been output by the methods
described in the previous section must be openly available. "Upon
request" availability is not acceptable (but online restricted access
only in exceptional cases is) and it is up to the authors to ensure that
proper permission is granted by participants (if the study includes data
obtained in this manner). Materials as intended here include, but are
not limited to, original audio or video recordings, transcripts,
experimental code, stimuli, gathered data files, researchers' notebooks,
derived data, annotations, statistical code etc.

\section{Discussion}\label{discussion}

The discussion section, to be added at Stage 2, will include the
author's interpretation of the findings reported in the previous
section, in light of the research topic, questions and/or hypothesis
presented in the Introduction. We recommend to keep the discussion
section short and to the point, while ensuring that due weight is given
to the evidentiary level of the research findings (i.e.~any claim made
in this section should be of a magnitude that does not exceed the
magnitude of evidence and uncertainty that characterise the findings).
Suggested further subsections include ``Limitations'' and ``Future
work/Open questions''.

\section{Conclusion}\label{conclusion}

The Conclusion section should briefly summarise the research topic,
questions and/or hypotheses and highlight the main findings in light of
existing gaps. We recommend to write a conclusion that could be read as
stand-alone. It is acceptable for the conclusion to mirror the structure
of the abstract and be an edited and slightly extended version of the
abstract.

\section*{CRediT roles}\label{credit-roles}
\addcontentsline{toc}{section}{CRediT roles}

This section must include a list of all authors with Contributor Role
Taxonomy (CRediT) roles listed for each author. See
\url{https://credit.niso.org}. All authors in the paper must either have
a ``Writing -- original draft'' role or a ``Writing -- review \&
editing'' and all authors must have at least two roles.

\section*{Acknowledgements}\label{acknowledgements}
\addcontentsline{toc}{section}{Acknowledgements}

You can use this section to acknowledge feedback and support by external
individuals or institutions.

\section*{Open Research statement}\label{open-research-statement}
\addcontentsline{toc}{section}{Open Research statement}

You are required to publicly share the Research Compendium of the study,
which must include all materials like data, code, stimuli,
questionnaires, figures etc. All analyses must be computationally
reproducible (when applicable) or documented to the highest detail and
level of transparency. This section must contain at least the following
statement, where XXX is a link, a DOI, or a (data) citation:

\begin{quote}
The Research Compendium of the study is available at XXX.
\end{quote}

With a data citation.

\begin{quote}
The Research Compendium of the study is available as Coretta, Hampton \&
De Cia (2025).
\end{quote}

\section*{Conflict of interest}\label{conflict-of-interest}
\addcontentsline{toc}{section}{Conflict of interest}

You should report here any conflict of interest. If you have no conflict
of interest, write: The authors confirm that there are no conflicts of
interest.

\section*{References}\label{references}
\addcontentsline{toc}{section}{References}

\phantomsection\label{refs}
\begin{CSLReferences}{1}{0}
\bibitem[\citeproctext]{ref-alfarano2021}
Alfarano, Valentina. 2021. \emph{A grammar of {Nal{ö}go}, an {Oceanic}
language of {Santa Cruz}}. Institut National des Langues et
Civilisations Orientales (LACITO), Paris PhD thesis.

\bibitem[\citeproctext]{ref-coretta2025}
Coretta, Stefano, Jessica Hampton \& Simone De Cia. 2025. Vitality
assessment of {Gallo-Romance} of {Northern Italy} {[}{Research
Compendium} v1.0{]}. Zenodo.
\url{https://doi.org/10.5281/ZENODO.15715938} (22 June, 2025).

\bibitem[\citeproctext]{ref-croft2002}
Croft, William. 2002. \emph{Typology and universals}. Cambridge:
Cambridge University Press.
\url{https://doi.org/10.1002/9781119072256.ch3}.

\bibitem[\citeproctext]{ref-jakobson1951}
Jakobson, Roman, Gunnar Fant \& Morris Halle. 1951. \emph{Preliminaries
to speech analysis. {The} distinctive features and their correlates}.
MIT Press, Cambridge MA.

\bibitem[\citeproctext]{ref-lindsey2017}
Lindsey, Geoff. 2017. The vowel space.
\url{https://www.englishspeechservices.com/blog/the-vowel-space/}.

\end{CSLReferences}




\end{document}
